\chapter{Introduction}
%%---------------------------------------------------------
The introduction of the TFM should serve so that teachers who evaluate the work can understand the context in which it is carried out, and the objectives that are set.

This template shows the basic structure of the final TFM memory, as well as some formatting instructions.
The basic schema of a final  TFM memory is:
\begin{itemize}
\item[•] Summary in Spanish and English (max 2 pages each)
\item[•] Content Table
\item[•] Introduction (with TFM's goals)
\item[•] Master’s Project Content
\item[•] Results and conclusions
\item[•] Bibliography (publications used in the study and work development)
\item[•] Annex (optional)
\end{itemize}


In any case, it is the tutor of the TFM who will indicate to his student the final memory structure that best fits the work developed.

Regarding the format, the following guidelines will be followed, which are shown in this template:
\begin{itemize}
\item[•] \textit{Paper size:} DIN A4
\item[•] \textit{Cover Page:} as stated in this template, with indication of university, center, TFM title and author.
\item[•] \textit{Second page:} bibliographic information, including all data of the TFM tutor.
\item[•] \textit{Letter type for text.} Preferably "Bookman Old Style" 11 points. If this is not possible, the recommended alternatives are, in order of preference: "Palatino Linotype", "Garamond" or "Georgia".
\item[•] \textit{Letter type for source code:} “Consolas” or “Roboto mono”
\item[•] \textit{Margins:} upper and lower $3$ cm, left and $2.54$ cm right.
\item[•] \textit{Sections and subsections:} reviewed with decimal numbering after the chapter number. Ej.: subsections 2.3.1.
\item[•] \textit{Page numbers:} always centered on the lower margin, page 1 begins in chapter 1, all sections before chapter 1 in lowercase Roman numeral (i, ii, iii…).
\end{itemize}

\vspace*{1.5cm}
To prepare the final memory of the TFM with this template, follow the steps below:
\begin{enumerate}
\item Download and Install MiKTeX:  \url{https://miktex.org/}
\item Download and install a \LaTeX~ editor, for example Texmaker:\\
\url{https://www.xm1math.net/texmaker/}

\item Edit the file \textbf{secciones/ \_DatosTFM.tex}, which is included in the folder \textbf{secciones} of this template. Fill in all the requested data in said file. Save and close
\item Compile the file
 \textbf{plantilla\_TFM.tex} (can be renamed). A file \textbf{pdf} will be generated as a result.
\item To write the final memory of the TFM you can add and / or modify the files in the \textbf{secciones} as necessary. The result is obtained when compiling the file \textbf{plantilla\_TFM.tex}. 
\end{enumerate}


\section{Source Code example in Python}
\begin{lstlisting}[style=Python]
# -*- coding: utf-8 -*-
import sympy as sy
from sympy.abc import x
\end{lstlisting}
%%---------------------------------------------------------